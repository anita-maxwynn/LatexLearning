\documentclass[11pt]{article}
\usepackage{amsfonts,amssymb,amsmath}
\usepackage{float} % <-- Needed for [H] placement specifier in table
\begin{document}

Arrays and table

Epsilon in --- $\in$

$\mathbb{R}$ for the set of real numbers\\

$[a]$ for square brackets\\

$\{ \}$ for curly brackets — use `\` before both\\

$\$50$ to display a dollar symbol\\

Always wrap big expressions with `left` and `right` for auto-sized brackets\\

\[
\left( \frac{a + b}{c + d} \right)
\]

We use `left.` or `right.` if we don't want the other one to be shown\\

\[
\left. \frac{dy}{dx} \right|_{x=1}
\]

\begin{table}[H]
\centering % for centering the table
\caption{Hell yeah table}
\def\arraystretch{1.2}  
\begin{tabular}{|c|p{11cm}|c|c|} \hline % c means centered; repeat for each column p means paragraph
x & o hell nah & 23 & 25 \\ \hline
x & o hell nah & 23 & 25 \\ \hline
x & o hell nah & 23 & 25 \\ \hline % \hline for horizontal line
x & o hell nah o hell nah o hell nah o hell nah o hell nah o hell nah o hell nah o hell nah o hell nah o hell nah o hell nah o hell nah o hell nah o hell nah o hell nah o hell nah o hell nah o hell nah  & 23 & 25 \\ \hline % \hline for horizontal line
\end{tabular}
\end{table}


Arrays:
\begin{align}
5x^2-9=x+3\\
5x^2-x-12=0
\end{align}


\begin{align*}
5x^2-9&=x+3\\
5x^2-x-12&=0\\
&=12+x-5x^2
\end{align*}


\begin{align}
5x^2-9=x+3\\
5x^2-x-12=0
\end{align}






\end{document}
